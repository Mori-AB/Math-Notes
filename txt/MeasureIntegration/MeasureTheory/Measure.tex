\setsection{Measure}
\label{sec:measure}

Informally speaking, measure is an abstraction of volume.
Volume, in turn, can be viewed as a function \(v\) which assigns
a nonnegative value to each subspace \(E\) of some given space \(X\).
Normally, one might impose more conditions such as:
\begin{itemize}
    \item Every subspace \(E\) of \(X\) must have some area.
    \item The empty set must have volume zero.
    \item Given disjoint \(E,F\subset X\), \(v(E\sqcup F)=v(E)+v(F)\).
    In other words,
    the area of the disjoint union \(E\sqcup F\) of subspaces
    should equal the sum of each of the areas.
    \item If \(X=\Rmath^n\),
    then \(v\) must be invariant under rigid motion.
\end{itemize}
Experience from topology also suggests that
the whole space \(X\) need not have infinite volume;
since \(X\) might be just a subspace of some larger space \(Y\).

Does there exist some function \(v\)
which satisfies all the properties above?
Polish mathematicians Stefan Banach and Alfred Tarski suggested that
the answer is \emph{no}.
Their counterexample is called the Banach-Tarski paradox,
which is stated below.

\begin{example}[Banach-Tarski Paradox]
    \label{exm:banach-tarski}
    Consider the unit ball \(B^3\) and its surface \(S^2\)
    in \(\Rmath^3\).
    Then it is possible to decompose \(B^3\)
    into a finite number of subsets,
    and apply only translation and rotation
    to obtain two balls identical to \(B^3\).
\end{example}

Hence, we need a more refined version of volume.
It was Henri Lebesgue who laid the definitions that are used today.
One observation from \cref{exm:banach-tarski} is that
we cannot assign volume to all sets;
our approach is to assign volume whenever possible.

\begin{definition}[\(\sigma\)-algebras]
    \label{def:sigmaalg}
    Let \(S\) be a set.
    Then \(\Sigma\subset\power{S}\) is called
    an \define{algebra}
    (or a \define{set algebra},
    for contrast with algebra in abstract algebra)
    on \(S\) if
    \begin{axioms}[\(\sigma\)-ALG]
        \item \(S\in\Sigma\);
        \item \(\Sigma\) is closed under complements,
        i.e., for any \(A\in\Sigma\),
        its complement \(A\compl\in\Sigma\);
        \item for any \(A,B\in\Sigma\), \(A\cup B\in\Sigma\).
    \end{axioms}
    Moreover, an algebra \(\Sigma\) is called
    a \define{\(\sigma\)-algebra} on \(S\)
    if it is closed under \emph{countable} union,
    in other words,
    \begin{axioms}[\(\sigma\)-ALG]
        \setcounter{enumi}{3}
        \item for any \(\indexset{A_n}{n=1}{\infty}\subset\Sigma\),
        \(\bigcup_{n=1}^\infty A_n\in\Sigma\).
    \end{axioms}
\end{definition}

Of course, there exist two kinds of trivial \(\sigma\)-algebras:
\begin{example}
    \label{exm:trivialsigma}
    For any set \(S\),
    its power set \(\power{S}\)
    and \(\{\varnothing,S\}\) are (\(\sigma\)-) algebras on \(S\).
\end{example}
Here, \(\power{S}\) is the maximum (in the sense of set inclusions)
\(\sigma\)-algebra on \(S\),
where \(\{\varnothing,S\}\) is the minimum \(\sigma\)-algebra on \(S\).

A trivial but intuitively significant remark follows:
\begin{remark}
    \label{rem:sigmaalg}
    Let \(S\) be a set and \(\Sigma\) be an algebra on \(S\).
    Then it follows directly from the axioms of \cref{def:sigmaalg} that
    \(\varphi=S\setminus S\in\Sigma\)
    and
    \[
        A\cap B
        =(A\compl\cup B\compl)\compl
        \in\Sigma,
        \quad
        A\setminus B
        =A\cap B\compl
        \in\Sigma
    \]
    hold for any \(A,B\in\Sigma\).
    Thus, it inductively follows that
    given any \(A_1,\dots,A_n\in\Sigma\),
    any finite set operation on \(\indexset*{A_k}{k=1}{n}\)
    will still yield a member of \(\Sigma\).
    In other words,
    \(\Sigma\) is closed under finite set operations.
    
    Similarly, if \(\Sigma\) is a \(\sigma\)-algebra on \(S\),
    then for a countable family \(\indexset*{A_n}{n=1}{\infty}\)
    \[
        \bigcap_{n=1}^\infty A_n
        =\Bigl(\bigcup_{n=1}^\infty A_n\compl\Bigr)\compl
        \in\Sigma;
    \]
    thus \(\Sigma\) is closed under countable set operations.
\end{remark}

\begin{definition}[Measurable Spaces]
    \label{def:mblspace}
    Let \(S\) be a set
    and \(\Sigma\) be a \(\sigma\)-algebra on \(S\).
    Then the pair \((S,\Sigma)\) is called
    a \define{measurable space}.
    In this spirit,
    we call elements of \(\Sigma\) as
    \define[measurable set!\(\Sigma\)-]{\(\Sigma\)-measurable sets}.
\end{definition}

The name ``measurable space'' naturally suggests that
the space is to be measured later;
this is precisely what we are about to do.

As with topological spaces,
measure spaces are in principle not special at all;
rather, they can be generated quite easily.

\begin{definition}[Generated \(\sigma\)-algebras]
    \label{def:gensigmaalg}
    Let \(S\) be a set,
    and \(\mathcal C\subset\power{S}\) be some collection
    of subsets of \(S\).
    Then \(\sigma(\mathcal C)\) is defined as
    the smallest \(\sigma\)-algebra on \(S\)
    which contains \(\mathcal C\) as a subset.
\end{definition}

To establish the existence of such a \(\sigma\)-algebra,
it is convenient to observe the following
(as for various structures from before):
\begin{proposition}
    \label{prop:intersectsigma}
    Let \(S\) be a set,
    and \(\indexset*{\Sigma_\alpha}{\alpha\in I}{}\)
    be a collection of \(\sigma\)-algebras on \(S\).
    Then the set
    \[
        \Sigma
        =\bigcap_{\alpha\in I}\Sigma_\alpha
    \]
    is again a \(\sigma\)-algebra on \(S\).
\end{proposition}
\begin{myproof}
    We need to verify the three axioms of \cref{def:sigmaalg}
    for \(\Sigma\).
    \begin{itemize}
        \item Since \(S\in\Sigma_\alpha\) for all \(\alpha\),
        \(S\in\Sigma\).

        \item Suppose that \(A\in\Sigma_\alpha\) for all \(\alpha\).
        Then since \(A\compl\in\Sigma_\alpha\) for all \(\alpha\),
        \(A\compl\in\Sigma\).

        \item Suppose that \(\indexset*{A_n}{n=1}{\infty}\in\Sigma\).
        Then since the collection is in every \(\Sigma_\alpha\),
        their union is also in \(\Sigma_\alpha\);
        thus in \(\Sigma\).
        \rightqed
    \end{itemize}
\end{myproof}
Note that
unions of \(\sigma\)-algebras are generally not \(\sigma\)-algebras.
(For a quick example,
consider \(S=\{0,1,2\}\) and its two \(\sigma\)-algebras
\[
    \Sigma_1=\{\varnothing,\{1\},\{2,3\},S\},
    \quad
    \Sigma_2=\{\varnothing,\{2\},\{1,3\},S\}.
\]
Then their union \(\Sigma_1\cup\Sigma_2\) does not contain \(\{1,2\}\),
hence is not closed under set operations.)
We already experienced this phenomenon
for sums (or unions) of vector spaces and unions of topologies.

Having proven this,
we can define \(\sigma(\mathcal C)\) in \cref{def:gensigmaalg} as
\[
    \sigma(\mathcal C)
    :=\bigcap_{\substack{
        \Sigma:\ \text{\(\sigma\)-algebra on \(S\)} \\
        \mathcal C\subset\Sigma}}\Sigma,
\]
where the above is a nonempty union since
\(\power{S}\) is such a \(\sigma\)-algebra.
Note that
we did not impose any condition on what \(\mathcal C\) should be.
In other words,
if some subsets of \(S\) are needed in a measure theory-related context,
we simply gather those subsets in \(\mathcal C\)
and consider \(\sigma(\mathcal C)\) to make \(S\) a measurable space.
Thus, in practice, a measure-theoretic interpretation of a situation
boils down to choosing the appropriate \(\sigma\)-algebra on \(S\)
which best describes the situation.

In spirit of the above idea,
we present some examples of creating \(\sigma\)-algebras
from existing ones.
\begin{proposition}[New \(\sigma\)-algebras form Old]
    \label{prop:sigmaalg}
    Let \((S,\Sigma)\) be a measurable space.
    \begin{alist}
        \item (Subspace \(\sigma\)-algebras)
        Given a nonempty subset \(A\subset S\),
        the collection \(\Sigma_A=\set{E\cap A}{E\in\Sigma}\)
        is a \(\sigma\)-algebra on \(A\).
        \item (Inverse images)
        Let \(T\) be a set,
        and \(f:T\to S\) be an arbitrary function.
        Then the collection
        \(\inv f(\Sigma)=\set{\inv F(E)}{E\in\Sigma}\)
        is a \(\sigma\)-algebra on \(T\).
    \end{alist}
\end{proposition}
\begin{myproof}
    \begin{alist}
        \item We verify the three axioms of \cref{def:sigmaalg}
        for \(\Sigma_A\).
        It immediately follows that \(A=S\cap A\in\Sigma_A\).
        
        Suppose that
        \(E\cap A\) is in \(\Sigma_A\) for some \(E\in\Sigma\).
        Then since \(\Sigma\) is a \(\sigma\)-algebra on \(S\),
        \(E\compl\in\Sigma\) implies that
        \(A\setminus(E\cap A)=A\cap E\compl\in\Sigma_A\).
        Finally, suppose that
        \(\indexset{A\cap E_n}{n=1}{\infty}\subset\Sigma_A\)
        for some \(\indexset{E_n}{n=1}{\infty}\subset\Sigma\).
        Then since
        \[
            A\cap\bigcup_{n=1}^\infty E_n
            =\bigcup_{n=1}^\infty(A\cap E_n)
        \]
        by distributivity
        and \(\bigcup_{n=1}^\infty E_n\in\Sigma\)
        since \(\Sigma\) is a \(\sigma\)-algebra on \(S\),
        we conclude that
        \(\Sigma_A\) is closed under countable union.
        Therefore, \(\Sigma_E\) is indeed a \(\sigma\)-algebra on \(E\).
        
        \item We verify the three axioms of \cref{def:sigmaalg}
        for \(\inv f(\Sigma)\).
        It immediately follows that \(T=\inv f(S)\in\inv f(\Sigma)\).
        
        Suppose that
        \(\inv f(E)\) is in \(\inv f(\Sigma)\) for some \(E\in\Sigma\).
        Then since \(\Sigma\) is a \(\sigma\)-algebra on \(S\),
        \(S\setminus E\in\Sigma\) implies that
        \(T\setminus\inv f(E)=\inv f(S\setminus E)\in\inv f(\Sigma)\).
        Finally, suppose that
        \(\indexset{\inv f(E_n)}{n=1}{\infty}\subset\inv f(\Sigma)\)
        for some \(\indexset{E_n}{n=1}{\infty}\subset\Sigma\).
        Then since
        \[
            \bigcup_{n=1}^\infty\inv f(E_n)
            =\inv f\biggl(\bigcup_{n=1}^\infty E_n\biggr)
        \]
        and \(\bigcup_{n=1}^\infty E_n\in\Sigma\)
        since \(\Sigma\) is a \(\sigma\)-algebra on \(S\),
        we conclude that
        \(\inv f(\Sigma)\) is closed under countable union.
        Therefore,
        \(\inv f(\Sigma)\) is indeed a \(\sigma\)-algebra on \(T\).
        \rightqed
    \end{alist}
\end{myproof}

\begin{example}
    \label{exm:gensigmaalg}
    \begin{alist}
        \item For any subset \(E\) of \(S\),
        the \(\sigma\)-algebra \(\sigma(E)\) generated by \(E\)
        is precisely the set \(\{\varnothing,E,S\setminus E,S\}\).
        \item Similarly,
        given a finite collection \(\indexset*{E_k}{k=1}{n}\)
        of subsets of \(S\),
        \[
            \sigma(\indexset*{E_k}{k=1}{n})
            =\set{\tilde E_1\cap\dots\cap\tilde E_n}%
                {\tilde E_i\in\sigma(E_i)}.
        \]
        For explicit families of sets
        (in other words, those represented with braces),
        then we often omit either the braces or parentheses.
        Thus, it is also common practice to write
        \[
            \sigma(\indexset*{E_k}{k=1}{n})
            =\sigma\indexset*{E_k}{k=1}{n}
            =\sigma(E_k\given k=1,\dots,n).
        \]
        Moreover, if \(S\) is the disjoint union of \(E_1,\dots,E_n\),
        then
        \[
            \sigma\indexset*{E_k}{k=1}{n}
            =\set{\tilde E_1\cup\dots\cup\tilde E_n}%
                {\tilde E_i=E_i\ \text{or}\ \varnothing}.
        \]
    \end{alist}
\end{example}
One may easily come up with similar forms for \(\sigma\)-algebras
generated by a countable or arbitrary collection of subsets of \(S\).
However, as we discover later,
the moral of some parts of the theory is that
\(\sigma\)-algebras are hard to compute or manipulate;
thus we invent some tools to work with them later on.

For topological spaces, there exists a natural \(\sigma\)-algebra.

\begin{definition}[Borel \(\sigma\)-algebras]
    \label{def:borelsig}
    Let \((S,\mathcal T)\) be a topological space.
    Then the \(\sigma\)-algebra \(\sigma(\mathcal T)\)
    generated by the topology \(\mathcal T\) of \(S\)
    is called the \define{Borel \(\sigma\)-algebra} on \(S\),
    and its elements are said
    to be \define[measurable set!Borel-]{Borel measurable},
    or simply a \textdef{Borel set}.
    If the topology in discussion is clear,
    then we denote the Borel \(\sigma\)-algebra on \(S\)
    by \(\denote{\borel(S)}\).
\end{definition}

Note that
the Borel algebra \(\borel(S)\) contains a lot more information
than the topology itself.
By the second axiom of \cref{def:sigmaalg},
\(\borel(S)\) also contains closed subsets of \(S\).
Moreover, from the third axiom of \cref{def:sigmaalg}
we observe that countable intersection of open sets
as well as countable union of closed sets
are also members of \(\borel(S)\),
while they are usually not open or closed in the topological sense.

But why do we care about Borel \(\sigma\)-algebras?
One possible reason (that will later show up) is that
since it is generated by the topology of the space,
continuous functions from the space ``behave well''
with respect to Borel \(\sigma\)-algebras.
A more reasonable approach at this stage might be that
in the usual real line, it is natural to measure
the length of open, closed, half-open intervals;
these are typically Borel sets.

Of course, it remains to define what a length means.

\begin{definition}[\(\sigma\)-additive Maps]
    \label{def:addmap}
    Let \(\Sigma_0\) be a set algebra for \(S\).
    Then an extended real-valued set function
    \(\mu_0:\Sigma_0\to[0,\infty]\)
    is said to be \define{additive} if
    \begin{axioms}[ADD]
        \item \(\mu_0(\varnothing)=0\);
        \item for disjoint \(A,B\in\Sigma_0\),
        \(\mu_0(A\sqcup B)=\mu_0(A)+\mu_0(B)\).
    \end{axioms}
    Moreover, an additive map \(\mu_0:\Sigma_0\to[0,\infty]\)
    is said
    to be \define{\(\sigma\)-additive} (or \textdef{countably additive})
    if
    \begin{axioms}[ADD]
        \setcounter{enumi}{2}
        \item for any mutually disjoint collection
        \(\indexset*{A_n}{n=1}{\infty}\subset\Sigma_0\)
        such that \(\bigcup_{n=1}^\infty A_n\in\Sigma_0\),
        \[
            \mu_0\Bigl(\bigsqcup_{n=1}^\infty A_n\Bigr)
            =\sum_{n=1}^{\infty}\mu_0(A_n).
        \]
    \end{axioms}
\end{definition}

There are some subtleties in the definition of \(\sigma\)-additive maps.
\begin{itemize}
    \item \Cref{def:addmap} considers \emph{algebras},
    not \(\sigma\)-algebras;
    thus the condition \(\bigcup_{n=1}^\infty A_n\in\Sigma_0\)
    is required in axiom (ADD-3)
    or otherwise the value on the left may not be defined.
    \item It is crucial that
    an additive map \(\mu_0\) may have \(\infty\) as its value;
    this enables us to state flexible arguments regarding infinite sums,
    but at the cost that one must avoid situations
    where the expression \(\infty-\infty\) arises.
    For example,
    if \(\mu_0\) were to take only finite values,
    then we would have stated that
    (ADD-3) holds only if the series of the right converges.
\end{itemize}

It is now time for our main definition:
\begin{definition}[Measure Spaces]
    \label{def:measurespace}
    Let \((S,\Sigma)\) be a measurable space.
    If \(\mu:\Sigma\to[0,\infty]\) is a \(\sigma\)-additive map,
    then we call \(\mu\) a \define{measure} on \((S,\Sigma)\),
    and the triple \((S,\Sigma,\mu)\) to be a \define{measure space}.
\end{definition}