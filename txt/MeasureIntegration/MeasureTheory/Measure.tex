\setsection{Measure}
\label{sec:measure}

Informally speaking, measure is an abstraction of volume.
Volume, in turn, can be viewed as a function \(v\) which assigns
a nonnegative value to each subspace \(E\) of some given space \(X\).
Normally, one might impose more conditions such as:
\begin{itemize}
    \item Every subspace \(E\) of \(X\) must have some area.
    \item The empty set must have volume zero.
    \item Given disjoint \(E,F\subset X\), \(v(E\sqcup F)=v(E)+v(F)\).
    In other words,
    the area of the disjoint union \(E\sqcup F\) of subspaces
    should equal the sum of each of the areas.
    \item If \(X=\Rmath^n\),
    then \(v\) must be invariant under rigid motion.
\end{itemize}
Experience from topology also suggests that
the whole space \(X\) need not have infinite volume;
since \(X\) might be just a subspace of some larger space \(Y\).

Does there exist some function \(v\)
which satisfies all the properties above?
Polish mathematicians Stefan Banach and Alfred Tarski suggested that
the answer is \emph{no}.
Their counterexample is called the Banach-Tarski paradox,
which is stated below.

\begin{example}[Banach-Tarski Paradox]
    \label{exm:banach-tarski}
    Consider the unit ball \(B^3\) and its surface \(S^2\)
    in \(\Rmath^3\).
    Then it is possible to decompose \(B^3\)
    into a finite number of subsets,
    and apply only translation and rotation
    to obtain two balls identical to \(B^3\).
\end{example}

Hence, we need a more refined version of volume.
It was Henri Lebesgue who laid the definitions that are used today.
One observation from \cref{exm:banach-tarski} is that
we cannot assign volume to all sets;
our approach is to assign volume whenever possible.

\begin{definition}[\(\sigma\)-algebras]
    \label{def:sigmaalg}
    Let \(S\) be a set.
    Then \(\Sigma\subset\power{S}\) is called
    an \define{algebra}
    (or a \define{set algebra},
    for contrast with algebra in abstract algebra)
    on \(S\) if
    \begin{axioms}[\(\sigma\)-ALG]
        \item \(S\in\Sigma\);
        \item \(\Sigma\) is closed under complements,
        i.e., for any \(A\in\Sigma\),
        its complement \(A\compl\in\Sigma\);
        \item for any \(A,B\in\Sigma\), \(A\cup B\in\Sigma\).
    \end{axioms}
    Moreover, an algebra \(\Sigma\) is called
    a \define{\(\sigma\)-algebra} on \(S\)
    if it is closed under \emph{countable} union,
    in other words,
    \begin{axioms}[\(\sigma\)-ALG]
        \setcounter{enumi}{3}
        \item for any \(\indexset{A_i}{i=1}{\infty}\subset\Sigma\),
        \(\bigcup_{k=1}^\infty A_k\in\Sigma\).
    \end{axioms}
\end{definition}

Of course, there exist two kinds of trivial \(\sigma\)-algebras:
\begin{example}
    \label{exm:trivialsigma}
    For any set \(S\),
    its power set \(\power{S}\)
    and \(\{\varnothing,S\}\) are (\(\sigma\)-) algebras on \(S\).
\end{example}
Here, \(\power{S}\) is the maximum (in the sense of set inclusions)
\(\sigma\)-algebra on \(S\),
where \(\{\varnothing,S\}\) is the minimum \(\sigma\)-algebra on \(S\).

A trivial but intuitively significant remark follows:
\begin{remark}
    \label{rem:sigmaalg}
    Let \(S\) be a set and \(\Sigma\) be an algebra on \(S\).
    Then it follows directly from the axioms of \cref{def:sigmaalg} that
    \(\varphi=S\setminus S\in\Sigma\)
    and
    \[
        A\cap B
        =(A\compl\cup B\compl)\compl
        \in\Sigma,
        \quad
        A\setminus B
        =A\cap B\compl
        \in\Sigma
    \]
    hold for any \(A,B\in\Sigma\).
    Thus, it inductively follows that
    given any \(A_1,\dots,A_n\in\Sigma\),
    any finite set operation on \(\indexset*{A_k}{k=1}{n}\)
    will still yield a member of \(\Sigma\).
    In other words,
    \(\Sigma\) is closed under finite set operations.
    
    Similarly, if \(\Sigma\) is a \(\sigma\)-algebra on \(S\),
    then for a countable family \(\indexset*{A_n}{n=1}{\infty}\)
    \[
        \bigcap_{n=1}^\infty A_n
        =\Bigl(\bigcup_{n=1}^\infty A_n\compl\Bigr)\compl
        \in\Sigma;
    \]
    thus \(\Sigma\) is closed under countable set operations.
\end{remark}

\begin{definition}[Measurable Spaces]
    \label{def:mblspace}
    Let \(S\) be a set
    and \(\Sigma\) be a \(\sigma\)-algebra on \(S\).
    Then the pair \((S,\Sigma)\) is called
    a \define{measurable space}.
    In this spirit,
    we call elements of \(\Sigma\) as
    \define[set!\(\Sigma\)-measurable]{\(\Sigma\)-measurable sets}.
\end{definition}

The name ``measurable space'' naturally suggests that
the space is to be measured later;
this is precisely what we are about to do.

As with topological spaces,
measure spaces are in principle not special at all;
rather, they can be generated quite easily.

\begin{definition}[Generated \(\sigma\)-algebras]
    \label{def:gensigmaalg}
    Let \(S\) be a set,
    and \(\mathcal C\subset\power{S}\) be some collection
    of subsets of \(S\).
    Then \(\sigma(\mathcal C)\) is defined as
    the smallest \(\sigma\)-algebra on \(S\)
    which contains \(\mathcal C\) as a subset.
\end{definition}

To establish the existence of such a \(\sigma\)-algebra,
it is convenient to observe the following
(as for various structures from before):
\begin{proposition}
    \label{prop:intersectsigma}
    Let \(S\) be a set,
    and \(\indexset*{\Sigma_\alpha}{\alpha\in I}{}\)
    be a collection of \(\sigma\)-algebras on \(S\).
    Then the set
    \[
        \Sigma
        =\bigcap_{\alpha\in I}\Sigma_\alpha
    \]
    is again a \(\sigma\)-algebra on \(S\).
\end{proposition}
\begin{myproof}
    We need to verify the three axioms of \cref{def:sigmaalg}
    for \(\Sigma\).
    \begin{itemize}
        \item Since \(S\in\Sigma_\alpha\) for all \(\alpha\),
        \(S\in\Sigma\).

        \item Suppose that \(A\in\Sigma_\alpha\) for all \(\alpha\).
        Then since \(A\compl\in\Sigma_\alpha\) for all \(\alpha\),
        \(A\compl\in\Sigma\).

        \item Suppose that \(\indexset*{A_n}{n=1}{\infty}\in\Sigma\).
        Then since the collection is in every \(\Sigma_\alpha\),
        their union is also in \(\Sigma_\alpha\);
        thus in \(\Sigma\).
        \rightqed
    \end{itemize}
\end{myproof}

Having proven this,
we can define \(\sigma(\mathcal C)\) in \cref{def:gensigmaalg} as
\[
    \sigma(\mathcal C)
    :=\bigcap_{\substack{
        \Sigma:\ \text{\(\sigma\)-algebra on \(S\)} \\
        \mathcal C\subset\Sigma}}\Sigma,
\]
where the above is a nonempty union since
\(\power{S}\) is such a \(\sigma\)-algebra.
Note that
we did not impose any condition on what \(\mathcal C\) should be.
In other words,
if some subsets of \(S\) are needed in a measure theory-related context,
we simply gather those subsets in \(\mathcal C\)
and consider \(\sigma(\mathcal C)\) to make \(S\) a measurable space.
Thus, in practice it boils down to choose the \(\sigma\)-algebra in need.

For topological spaces, there exists a natural \(\sigma\)-algebra.

\begin{definition}
    \label{def:borelsig}
    Let \((S,\mathcal T)\) be a topological space.
    Then the \(\sigma\)-algebra generated by the topology \(\mathcal T\)
    is called the \define{Borel \(\sigma\)-algebra} on \(S\).
\end{definition}