\setsection{Measure}
\label{sec:measure}

Informally speaking, measure is an abstraction of volume.
Volume, in turn, can be viewed as a function \(v\) which assigns
a nonnegative value to each subspace \(E\) of some given space \(X\).
Normally, one might impose more conditions such as:
\begin{itemize}
    \item Every subspace \(E\) of \(X\) must have some area.
    \item The empty set must have volume zero.
    \item Given disjoint \(E,F\subset X\), \(v(E\sqcup F)=v(E)+v(F)\).
    In other words,
    the area of the disjoint union \(E\sqcup F\) of subspaces
    should equal the sum of each of the areas.
    \item If \(X=\Rmath^n\),
    then \(v\) must be invariant under rigid motion.
\end{itemize}
Experience from topology also suggests that
the whole space \(X\) need not have infinite volume;
since \(X\) might be just a subspace of some larger space \(Y\).

Does there exist some function \(v\)
which satisfies all the properties above?
Polish mathematicians Stefan Banach and Alfred Tarski suggested that
the answer is \emph{no}.
Their counterexample is called the Banach-Tarski paradox,
which is stated below.

\begin{example}[Banach-Tarski Paradox]
    \label{exm:banach-tarski}
    Consider the unit ball \(B^3\) and its surface \(S^2\)
    in \(\Rmath^3\).
    Then it is possible to decompose \(B^3\)
    into a finite number of subsets,
    and apply only translation and rotation
    to obtain two balls identical to \(B^3\).
\end{example}