\setchapter{Measure Theory and Lebesgue Integration}
\label{chap:lebesgue}

Riemann's theory of integration on \(\Rmath\) is usually depicted as
partitioning the interval and obtaining long vertical strips.
Lebesgue's theory takes the other way round;
loosely speaking, his theory of integration involves the process of
making horizontal strips under the graph of the given function.
What seems miraculous is the fact that
a simple change of direction resolves a lot of irregularities
which appear in Riemann's integration.

But what do we actually mean by cutting the area horizontally?
To answer the question, we need to introduce the concept of measure.

\subimport{MeasureTheory/}{Measure.tex}