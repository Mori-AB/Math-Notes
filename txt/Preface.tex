\setchapter*{Preface}

This is a collection of personal study notes
which are taken during class, private studying, etc.
An ideal math textbook would consist of various concrete examples
with definitions and theorems which conveniently capture
the nature of these examples.
In this sense, this note is far from being ideal;
descriptions might not be neatly organized
and definitions might be wrong at times.
Noticing those imperfections and improving them on a regular basis
would be a motivating challenge for studying.

Exercises would mainly come from English, Korean and Japanese books,
since those are the languages that I comprehend.
For citations and quotes here and there,
references from other languages may also be included.
See the bibliography for the full list of references.

One would ideally partition books into classical subjects such as
analysis, linear algebra, complex analysis, topology and so on.
The issue I had with this approach is that
some topics from textbooks are repeated over and over,
while other topics are glossed away despite their complexity.
Needless to say,
topics logically cannot use ideas from their succeeding topics;
rather, results from mathematical pedagogy strongly suggests that
previous topics should be motivating ideas
for the abstraction that follows.

As a result, I have grouped some topics by the following list.
These are basically almost everything that I have learned.
Obviously, the order or content are subject to change;
however, perhaps the main lines would remain quite unchanged.

\begin{itemize}
    \item \textdef{Logic and Set Theory}
    \begin{itemize}
        \item Logic, Set Theory
        \item Construction of Numbers, Axiomatic Geometry
    \end{itemize}
    \item \textdef{Basic Algebra}
    \begin{itemize}
        \item Algebraic Structures, Polynomial Algebra
        \item Matrix Algebra, Linear Algebra
        \item Tensors
    \end{itemize}
    \item \textdef{Point-set Topology}
    \begin{itemize}
        \item Topological Spaces
        \item Properties of Spaces
        \item Construction of Spaces
        \item Separation Axioms
    \end{itemize}
    \item \textdef{Functions of a Real Variable}
    \begin{itemize}
        \item Metric Spaces
        \item Limits and Continuity
        \item Function Spaces
        \item Derivatives and Analytic Functions
    \end{itemize}
    \item \textdef{Measure Theory and Calculus}
    \begin{itemize}
        \item Riemann-Stieltjes Integration
        \item Measure Theory
        \item Multivariable Calculus
        \item Integration of Differential Forms
    \end{itemize}
    \item \textdef{Functions of a Complex Variable}
    \begin{itemize}
        \item Holomorphic and Meromorphic Functions
        \item Conformal Mappings
        \item Elliptic Functions and Modular Forms
    \end{itemize}
    \item \textdef{Algebra}
    \begin{itemize}
        \item Category Theory
        \item Group Theory
        \item Field Theory
        \item Galois Theory
    \end{itemize}
    \item \textdef{Geometry on Manifolds}
    \begin{itemize}
        \item Structure of Manifolds
        \item Bundles and Fibered Manifolds
        \item Riemannian Manifolds
        \item Lie Groups and Lie Algebras
    \end{itemize}
    \item \textdef{Algebraic Topology}
    \begin{itemize}
        \item Homotopy and Covering Spaces
        \item Homology and Cohomology
    \end{itemize}
    \item \textdef{Topological Vector Spaces}
    \begin{itemize}
        \item Normed Spaces and Banach Spaces
        \item Inner Product Spaces and Hilbert Spaces
        \item \(C^\ast\)-algebra
        \item Distribution Theory
    \end{itemize}
    \item \textdef{Measure Theory and Random Variables}
    \begin{itemize}
        \item Probability as Measures
        \item Random Variables and Correlation
        \item Probability Distribution, Central Limit Theorem
    \end{itemize}
    \item \textdef{Differential Equations}
    \begin{itemize}
        \item Ordinary, Partial, Stochastic Differential Equations
        \item Calculus of Variations
        \item Numeric Approaches to Differential Equations
    \end{itemize}
\end{itemize}

A diagram showing interdependence (in terms of logic and ideas)
is drawn in \cref{fig:overallmap}.
The diagram itself may look messy,
but clearly indicates subjects of great importance.
For example, measure theory and calculus are no more or less than
one of the main paths which modern mathematics have emerged;
although there are some requirements for the theory,
the results are of absolute importance.
The same comment also holds for manifolds.
\begin{figure}[h!]
    \centering
    \begin{tikzpicture}
        \node (set) {Logic and Set Theory};
        \node (bsc) [below=1cm of set]{Basic Algebra};
        \node (tpo) [right=1cm of bsc]{Point-set Topology};
        \node (anl) [right=1cm of tpo]{Functions of a Real Variable};
        \node (int) [below=1cm of anl]{Measure Theory and Calculus};
        \node (cmp) [below=0.7cm of int]{Functions of a Complex Variable};
        \node (alg) [below=1cm of bsc]{Algebra};
        \node (mfd) [below=3.5cm of tpo]{Geometry on Manifolds};
        \node (atp) [below=1cm of alg]{Algebraic Topology};
        \node (tvs) [below=1.3cm of cmp]{Topological Vector Spaces};
        \node (rnd) [below=1cm of mfd]
            {Measure Theory and Random Variables};
        \node (deq) [below=2cm of tvs]{Differential Equations};
        
        \draw[->] (set.south) -- (bsc.north);
        \draw[->] (set.south) -- (tpo.north);
        \draw[->] (set.south) -- (anl.north);
        \draw[->] (tpo.east) -- (anl.west);
        \draw[->] (bsc.east) -- (int.west);
        \draw[->] (tpo.south east) -- (int.north west);
        \draw[->] (anl.south) -- (int.north);
        \draw[->] (anl.south east)
            .. controls ++ (1.2cm,-2.5cm) and ++ (0,7mm)
            .. (cmp.north);
        \draw[->] (bsc.south) -- (alg.north);
        \draw[->] (tpo.south) -- (mfd.north);
        \draw[->] (cmp.south) -- (mfd.east);
        \draw[->] (int.south west) -- (mfd.north east);
        \draw[->] (alg.south west)
            .. controls ++ (-2cm,-2cm) and ++ (-3cm,0)
            .. (mfd.west);
        \draw[->] (tpo.south west) -- (atp.north east);
        \draw[->] (alg.south) -- (atp.north);
        \draw[->] (bsc.south east) -- (tvs.north);
        \draw[->] (alg.east)
            .. controls ++ (3cm,0) and ++ (-1cm,1cm)
            .. (tvs.north west);
        \draw[->] (int.south east)
            .. controls ++ (7mm,-7mm) and ++ (7mm,1cm)
            .. (tvs.north east);
        \draw[->] (mfd.south) -- (rnd.north);
        \draw[->] (int.south)
            .. controls ++ (-3cm,-1cm) and ++ (-2cm,2cm)
            .. (rnd.north west);
        \draw[->] (tvs.south) -- (deq.north);
        \draw[->] (rnd.south) -- (deq.north west);
    \end{tikzpicture}
    \caption{An overall map of topics in these study notes}
    \label{fig:overallmap}
\end{figure}