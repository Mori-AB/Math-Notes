\setsection{The Definition of a Norm}
\label{sec:norm}

\begin{convention}
    \label{cnv:tvsfields}
    We focus on vector spaces over \(\Rmath\) or \(\Cmath\).
    If the same statement applies to both fields,
    we call the scalar field \(F\).
\end{convention}

This convention is rather practical;
after all, the study of topological vector spaces
(and ``functional'' analysis)
has started from the need in differential equations and such
to manipulate real or complex valued functions.

\begin{definition}[(Semi-)Norm]
    \label{def:norm}
    Let \(V\) be an \(F\)-vector space,
    and consider the map \(p:V\to\Rmath\).
    We call \(p\) a \define{norm} on \(V\)
    if the following holds for any \(v,w\in V\) and \(a\in F\):
    \begin{axioms}[Nm]
        \item \(p(av)=\abs{a}p(v)\);
        \hfill (Absolute homogenity)
        \item \(p(v+w)\le p(v)+p(w)\);
        \hfill (Subadditivity)
        \item \(p(v)=0\) implies \(v=0\).
    \end{axioms}
    If we drop condition (Nm-3),
    then we call \(p\) a \define{seminorm} on \(V\).
\end{definition}

Mathematicians claim that
norms are the abstraction of the daily notion of ``size.''
It \emph{commutes} with scaling,
in other words,
the size of the scaled object is the scaled size of the object,
and obeys the triangle inequality.
But it seems to miss a crucial point:
is it really nonnegative all the time?

\begin{observation}
    \label{obv:seminorm}
    Let \(p\) be a seminorm for an \(F\)-vector space \(V\).
    Then the following holds:
    \begin{alist}
        \item \(p(0)=0\).
        \item \(p(v)\ge 0\) for any \(v\in V\).
    \end{alist}
\end{observation}
\begin{myproof}
    \begin{alist}
        \item By homogenity of \(p\), we have \(p(0)=0p(0)=0\).
        \item Since \(p(-v)=\abs{-1}p(v)=p(v)\) for any \(v\in V\),
        subadditivity of \(p\) implies
        \[
            0
            =p(0)
            \le p(v)+p(-v)
            =2p(v);
        \]
        thus \(p(v)\ge 0\).
        \rightqed
    \end{alist}
\end{myproof}

\begin{definition}[Normed Space]
    \label{def:nspace}
    If an \(F\)-vector space \(V\) is given a norm \(p\),
    we call the pair \((V,p)\) a \define[space!normed]{normed space}.
    
    In particular, a norm \(p\) is often denoted by double vertical bars
    \(\denote{\norm{}{}}.\)
    In this case, we call \((V,\norm{}{})\) a normed space.
\end{definition}

As in the case with topological or metric spaces,
it is common practice to omit the norm
and simply call \(V\) a normed space.
Also, many notations for norms use double vertical bars
and list additional information on subscripts.

Some familiar examples of norms from analysis are listed below.

\begin{proposition}
    \label{prop:pnorm}
    \begin{alist}
        \item Consider the vector space \(F^n\).
        Then for any \(1\le p\),
        we define the \textdef{\(p\)-norm} on \(F^n\) by
        \[
            \norm{x}{p}
            =(\abs{x^1}^p+\cdots+\abs{x^n}^p)^{1/p}.
            \qquad
            (x=(x^1,\dots,x^n))
        \]
        Note that
        the Euclidean norm is simply the \(2\)-norm \(\norm{}{2}\),
        while the taxicab norm is the \(1\)-norm \(\norm{}{1}\).
        Moreover, by defining the ``\(\infty\)-norm''
        \[
            \norminf{x}
            =\max_{1\le k\le n}\abs{x^k}
            \qquad
            (x=(x^1,\dots,x^n))
        \]
        we can also cover the case \(p=\infty\),
        in the sense that
        \(\norm{x}{p}\) converges to \(\norminf{x}\) as \(p\to\infty\).
        
        \item Let \(F^\infty\) be the set of finite sequences in \(F\).
        Then for any \(1\le p\),
        we define the \textdef{\(p\)-norm} of \(\seq{a_n}{n}\) by
        \[
            \norm{\seq*{a_n}{n}}{p}
            =\biggl(\sum_{n=1}^{\infty}\abs{a_n}^p\biggr)^{1/p}.
        \]
        Note that \(\seq*{a_n}{n}\) is a finite sequence;
        thus the above is still a finite sum.
    \end{alist}
\end{proposition}

\begin{proposition}
    \label{prop:pnormint}
    \begin{alist}
        \item Consider the vector space \(F^n\).
        Then for any \(1\le p\),
        we define the \textdef{\(p\)-norm} on \(F^n\) by
        \[
            \norm{x}{p}
            =(\abs{x^1}^p+\cdots+\abs{x^n}^p)^{1/p}.
            \qquad
            (x=(x^1,\dots,x^n))
        \]
        Note that
        the Euclidean norm is simply the \(2\)-norm \(\norm{}{2}\),
        while the taxicab norm is the \(1\)-norm \(\norm{}{1}\).
        Moreover, by defining the ``\(\infty\)-norm''
        \[
            \norminf{x}
            =\max_{1\le k\le n}\abs{x^k}
            \qquad
            (x=(x^1,\dots,x^n))
        \]
        we can also cover the case \(p=\infty\),
        in the sense that
        \(\norm{x}{p}\) converges to \(\norminf{x}\) as \(p\to\infty\).
        
        \item Let \(F^\infty\) be the set of finite sequences in \(F\).
        Then for any \(1\le p\),
        we define the \textdef{\(p\)-norm} of \(\seq{a_n}{n}\) by
        \[
            \norm{\seq*{a_n}{n}}{p}
            =\biggl(\sum_{n=1}^{\infty}\abs{a_n}^p\biggr)^{1/p}.
        \]
        Note that \(\seq*{a_n}{n}\) is a finite sequence;
        thus the above is still a finite sum.
    \end{alist}
\end{proposition}