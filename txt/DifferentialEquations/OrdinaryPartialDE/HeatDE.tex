\setsection{The Heat Equation}
\label{sec:heat}

The heat equation is a partial differential equation
that describes the temperature of an object throughout time.
We start with the most basic example.

\begin{example}
    \label{exm:heat}
    Consider a thin, uniform rod of length \(l\).
    Let \(u=u(x,t)\) denote the temperature
    at the point \(x\in[0,l]\) and time \(t\ge 0\).
    
    Consider an interior point \(z\in(0,l)\) of the rod,
    and let \(z\in[\alpha,\beta]\subset[0,1]\).
    Then the heat energy \(Q(t)\) in the small interval is defined as
    \[
        Q(t)
        =\int_{\alpha}^{\beta}c\rho u(x,t)\dd x.
    \]
    Also, letting \(q=q(x,t)\) be the rate of heat flow
    towards the \(+x\) direction yields
    \[
        Q_t(t)
        =-q(\beta,t)+q(\alpha,t)+\int_{\alpha}^{\beta}\phi(x,t)\dd x,
    \]
    where \(\phi\) denotes the rate of 
    heat generation of our rod per volume.
    Assuming that \(u_t\) is continuous on \([0,l]\times[0,\infty)\),
    by the Leibniz integral rule
    and the fundamental theorem of calculus we obtain
    \begin{gather*}
        c\rho\int_{\alpha}^{\beta}\ptdrv{u}{t}\dd x
        =-q(\beta,t)+q(\alpha,t)+\int_{\alpha}^{\beta}\phi(x,t)\dd x
        =-\int_{\alpha}^{\beta}\ptdrv{q}{x}-\phi(x,t)\dd x, \\
        \frac{c\rho}{\beta-\alpha}
            \int_{\alpha}^{\beta}\ptdrv{u}{t}\dd x
        =-\frac{1}{\beta-\alpha}
            \int_{\alpha}^{\beta}\ptdrv{q}{x}-\phi(x,t)\dd x \\
        \intertext{and hence}
        c\rho\eval*{\ptdrv{u}{t}}{(z,t)}
        =-\eval*{\ptdrv{q}{x}}{(z,t)}+\phi(z,t);
    \end{gather*}
    where the last equation was obtained by taking the limit
    \(\alpha,\beta\to z\).

    To formulate the example as a problem for \(u\),
    we apply Fourier's law of conduction \(q=-\kappa u_x\).
    Thus we obtain
    \[
        c\rho\eval*{\ptdrv{u}{t}}{(z,t)}
        =\kappa\eval*{\ptdrv[2]{u}{x}}{(z,t)}+\phi(z,t),
    \]
    or in terms of functions,
    \begin{equation}
        \label{eq:heat}
        \ptdrv{u}{t}
        =\frac{\kappa}{c\rho}\ptdrv{q}{x}+\frac{1}{c\rho}\phi.
    \end{equation}
\end{example}

\begin{definition*}[The Heat Equation]
    \label{def*:heat}
    The differential equation given in \eqref{eq:heat}
    is called the \define{heat equation} of a finite 1D case.
    In particular, if \(\phi=0\),
    then the equation is called to be
    a \define[heat equation!homogeneous]{homogeneous heat equation}.
\end{definition*}

However, problems regarding differential equations
are never complete without initial and boundary conditionns.
There are two ways to define them.

\begin{definition*}[Initial and Boundary Conditions]
    \label{def*:IBVP}
    First suppose that
    the solution \(u\) for the differential equation
    \eqref{eq:heat} satisfies \(u(x,0)=a(x)\)
    for some known function \(a\).
    Then \(a\) is called the \define{initial condition} of \(u\).

    As for boundary values, suppose that
    the behavior \(u(0,t)\) and \(u(l,t)\) of \(u\)
    at the boundary of the rod are known functions \(b_0\) and \(b_1\).
    Such a condition is called
    the \define{Dirichlet boundary condition} of \(u\).
    On the other hand,
    suppose that the behavior of \(u_x\) instead of \(u\) are known,
    i.e.
    \[
        -\kappa\eval*{\ptdrv{u}{x}}{(0,t)}=b_0(t),
        \qquad
        -\kappa\eval*{\ptdrv{u}{x}}{(l,t)}=b_1(t).
    \]
    Such a condition is called
    the \define{Neumann boundary condition} of \(u\).
\end{definition*}

The formal numbered definitions will appear soon
when we consider heat equations of higher dimensional cases.
For now, we concentrate on the one-dimensional case.
Let us first state our simple problem
where \(l=1\) and \(\phi=0\).

\begin{question}
    \label{qst:homo1dheat}
    Does there exist a continuous function \(u=u(x,t)\)
    which satisfies
    \begin{align}
        \ptdrv{u}{t}&=\ptdrv[2]{u}{x},
        && (t>0,\ 0<x<1) \\
        u(x,0)&=a(x),
        && (0\le x\le 1) \\
        u(0,t)=b_0(x),
        &\qquad
        u(1,t)=b_1(x)?
        && (t>0)
    \end{align}
    given appropriate functions \(a\), \(b_0\) and \(b_1\)?
\end{question}

However, it is hard to consider an infinite region
(mostly because there might not exist a maximum value).
Moreover, most applications of partial differential equations
are interested in finding \(u(x,T)\) given some finite \(T>0\).
To do this, consider the domains
\[
    Q_T=(0,1)\times(0,T]
    \qquad\text{and}\qquad
    \Gamma_T=\closure{Q_T}\setminus Q_T.
\]
In other words,
\(Q_T\) is the region where the value of \(u\) is unknown,
whereas \(\Gamma_T\) is the region where \(u\) is known.