\setsection{Bilinear Maps}
\label{sec:bilinear}

\begin{example}
    \label{exm:bilinear}
    Let \(V\) be a vector space over a field \(k\),
    and \(V^*\) be its dual space.
    Then the map \(\epsilon:V\times V^*\to k\) defined by
    \((v,f)\mapsto f(v)\)
    is linear in both coordinates.
    In other words, given
    \(v,v_1,v_2\in V\), \(f,f_1,f_2\in V^*\) and \(a\in k\),
    the relations
    \[
        \epsilon(av_1+v_2,f)
        =a\epsilon(v_1,f)+\epsilon(v_2,f),
        \qquad
        \epsilon(v,af_1+f_2)
        =a\epsilon(v,f_1)+\epsilon(v,f_2)
    \]
    hold.
\end{example}
\begin{example}
    \label{exm:bilinear2}
    Let \(V=k^n\) for a field \(k\),
    and consider the map \(i:V\times V\to k\) given by
    \[
        (v,w)\mapsto\sum_{k=1}^nv_kw_k,
    \]
    where \(v_k\) and \(w_k\) are the coordinates of \(v\) and \(w\).
    This map is also linear in both coordinates; indeed,
    \[
        i(au+v,w)
        =\sum_{k=1}^{n}(au+v)_kw_k
        =a\sum_{k=1}^{n}u_kw_k+\sum_{k=1}^{n}v_kw_k
        =ai(u,w)+i(v,w)
    \]
    indicates that \(i\) is linear in the first coordinate;
    the case for the second coordinate is similar.
\end{example}

The above examples demonstrate a common example:
given a field \(k\), let \(V\) and \(W\) be
(not necessarily distinct) \(k\)-vector spaces,
and consider a map \(\lambda:V\times W\to k\).
We have observed that
\(V\times W\) is itself a vector space;
but at the moment, \(\lambda\) is not a linear map from \(V\times W\).
Instead, \(\lambda\) is linear only if
either \(v\in V\) or \(w\in W\) of the input is fixed.
Since \(\lambda\) is not linear,
our basic knowledge of linear algebra may not be suitable
to manipulate these kinds of objects.

\begin{definition}[Bilinear Maps and Forms]
    \label{def:bilinear}
    Suppose that \(V\), \(W\) and \(X\) are \(k\)-vector spaces,
    and \(\lambda:V\times W\to X\) be a map.
    If \(\lambda\) is linear on each of its coordinates, i.e.
    \begin{axioms}[BL]
        \item For any \(w\in W\),
        the map \(V\to k\) given by \(v\mapsto\lambda(v,w)\)
        is a linear map of \(V\);
        \item For any \(v\in V\),
        the map \(W\to k\) given by \(w\mapsto\lambda(v,w)\)
        is a linear map of \(W\),
    \end{axioms}
    then \(\lambda\) is called
    a \define[bilinear map]{\(k\)-bilinear map} of \(V\) and \(W\).
    In particular, if \(V=W\) and \(X=k\),
    then \(\lambda\) is called
    a \define[bilinear form]{\(k\)-bilinear form} on \(V\).
\end{definition}

The reason why we set the codomain of \(\lambda\)
to be another \(k\)-vector space instead of \(k\)
as in the previous examples
is illustrated in the next example.

\begin{example}
    \label{exm:bilinear3}
    Consider \(V=W=X=\Rmath^3\).
    Then the \textdef{cross product} in \(\Rmath^3\) defined by
    \[
        \trp*{a_1,a_2,a_3}\times\trp*{b_1,b_2,b_3}
        =\trp*{a_2b_3-a_3b_2,a_3b_1-a_1b_3,a_1b_2-a_2b_1}
    \]
    is a bilinear map.
    Indeed, taking the cross product of
    \(\trp*{ca_1+a_1',ca_2+a_2',ca_3+a_3'}\) and \(\trp*{b_1,b_2,b_3}\)
    we obtain
    \begin{align*}
        &\trp*{ca_1+a_1',ca_2+a_2',ca_3+a_3'}\times\trp*{b_1,b_2,b_3} \\
        ={}&\trp*{(ca_2+a_2')b_3-(ca_3+a_3')b_2,
            (ca_3+a_3')b_1-(ca_1+a_1')b_3,
            (ca_1+a_1')b_2-(ca_2+a_2')b_1} \\
        ={}&\trp*{ca_2b_3-ca_3b_2,ca_3b_1-ca_1b_3,ca_1b_2-ca_2b_1}
            +\trp*{a_2'b_3-a_3'b_2,a_3'b_1-a_1'b_3,a_1'b_2-a_2'b_1} \\
        ={}&c\trp*{a_1,a_2,a_3}\times\trp*{b_1,b_2,b_3}
            +\trp*{a_1',a_2',a_3'}\times\trp*{b_1,b_2,b_3}
    \end{align*}
    by direct computation.
\end{example}

As for the case with linear maps,
the collection of bilinear forms is also a vector space.

\begin{observation}
    \label{obv:blnvsp}
    Given \(k\)-vector spaces \(V\), \(W\) and \(X\),
    let \(\denote{\Lsp(V,W;X)}\) be
    the set of bilinear maps of \(V\) and \(W\) to \(X\).
    Then \(\Lsp(V,W;X)\) is a vector space under the operations
    \[
        (\lambda+\mu)(v,w)
        =\lambda(v,w)+\mu(v,w),
        \qquad
        (c\lambda)(v,w)
        =c\lambda(v,w),
    \]
    where \(v\in V\), \(w\in W\) and \(c\in k\).
\end{observation}
\begin{myproof}
    Let \(\lambda,\mu:V\times W\to X\) be bilinear forms,
    and \(c\in k\).
    We need to prove that
    \(\lambda+\mu\) and \(c\lambda\) are still bilinear forms.
    This follows from simple calculations,
    since for any vectors \(v_1,v_2\in V\), \(w\in W\)
    and \(d\in k\) we have
    \begin{align*}
        (\lambda+\mu)(cv_1+v_2,w)
        &=\lambda(cv_1+v_2,w)+\mu(cv_1+v_2,w) \\
        &=c\lambda(v_1,w)+\lambda(v_2,w)+c\mu(v_1,w)+\mu(v_2,w) \\
        &=c(\lambda+\mu)(v_1,w)+(\lambda+\mu)(v_2,w),
    \end{align*}
    and a similar result holds for the second coordinate.
\end{myproof}

In the case of finite dimensional \(k\)-vector spaces \(V\) and \(W\),
there is a simple way of constructing a bilinear map.

\begin{observation}
    \label{obv:blnmat}
    Let \(V\) and \(W\) be \(k\)-vector spaces
    with \(\dim V=n\) and \(\dim W=m\),
    and let \(\mathfrak B\) and \(\mathfrak C\) be respective bases
    of \(V\) and \(W\).
    Then given any matrix \(B\in\Msp_{m,n}(k)\),
    the map \(\lambda:V\times W\to k\) given by
    \[
        (v,w)\mapsto\trp*{\coord{w}{\mathfrak C}}B\coord{v}{\mathfrak B}
    \]
    is a bilinear map of \(V\) and \(W\).
\end{observation}
\begin{myproof}
    This follows from the fact that
    transposition and the coordinate maps
    \(v\mapsto\coord{v}{\mathfrak B}\),
    \(w\mapsto\coord{w}{\mathfrak C}\) are linear,
    and distribution holds for matrix operations.
    In other words,
    given any \(v_1,v_2\in V\), \(w\in W\) and \(c\in k\) we have
    \begin{gather*}
        \lambda(v_1+w_2,w)
        =\trp*{\coord{w}{\mathfrak C}}B\coord{v_1+v_2}{\mathfrak B}
        =\trp*{\coord{w}{\mathfrak C}}B\coord{v_1}{\mathfrak B}
            +\trp*{\coord{w}{\mathfrak C}}B\coord{v_2}{\mathfrak B}
        =\lambda(v_1,w)+\lambda(v_2,w), \\
        \lambda(cv_1,w)
        =\trp*{\coord{w}{\mathfrak C}}B\coord{cv_1}{\mathfrak B}
        =c\trp*{\coord{w}{\mathfrak C}}B\coord{v_1}{\mathfrak B}
        =c\lambda(v_1,w)
    \end{gather*}
    and a similar result holds for the second coordinate.
\end{myproof}

It seems that \cref{obv:blnmat} provides a nice method
to construct some bilinear maps to the base field \(k\).
A natural question arises:
are these the only bilinear maps to \(k\)?

\begin{theorem}
    \label{thm:blnrep}
    Let \(V\) and \(W\) be \(k\)-vector spaces
    with \(\dim V=n\) and \(\dim W=m\),
    and let \(\mathfrak B\) and \(\mathfrak C\) be respective bases
    of \(V\) and \(W\).
    Then for any bilinear map \(\lambda\in\Lsp(V,W;k)\),
    there uniquely exists a matrix \(B\in\Msp_{m,n}(k)\) such that
    \[
        \lambda(v,w)
        =\trp*{\coord{w}{\mathfrak C}}B\coord{v}{\mathfrak B}
    \]
    for all \(v\in V\) and \(w\in W\).
    Moreover,
    \(\Lsp(V,W;k)\) and \(\Msp_{m,n}(k)\) are isomorphic vector spaces.
\end{theorem}
\begin{sketch}
    Recall that
    linear maps are completely determined by their values
    at the basis elements.
    Since \(\lambda\) is linear on both of its variables,
    we can apply a similar argument.
\end{sketch}
\begin{myproof}
    Denote the elements of the given bases by
    \(\mathfrak B=\indexset*{v_j}{j=1}{n}\)
    and \(\mathfrak C=\indexset*{w_i}{i=1}{m}\).
    
    Now define \(b_{ij}=\lambda(v_j,w_i)\)
    for each \(i=1,\dots,m\) and \(j=1,\dots,n\);
    we claim that \(B=(b_{ij})\in\Msp_{m,n}(k)\) is the desired matrix.
    Verification is straightforward,
    taking \(v=\sum a_jv_j\) and \(w=\sum c_iw_i\) we obtain
    \begin{align*}
        \lambda(v,w)
        =\sum_{j=1}^{n}\sum_{i=1}^{m}a_jc_i\lambda(v_j,w_i)
        =\sum_{i=1}^{m}c_i\sum_{j=1}^{n}b_{ij}a_j
        =\sum_{i=1}^{m}c_i\row*{B\coord{v}{\mathfrak B}}{i}
        =\trp*{\coord{w}{\mathfrak C}}B\coord{v}{\mathfrak B},
    \end{align*}
    since \(\coord{v}{\mathfrak B}=\trp*{a_1,\dots,a_n}\)
    and \(\coord{w}{\mathfrak C}=\trp*{c_1,\dots,c_m}\).
    Uniqueness follows from our construction;
    since the \((i,j)\)-component of \(B\)
    \emph{must} be \(\lambda(v_j,w_i)\).

    To prove the second claim, it suffices to show that
    our construction preserves addition and scalar multiplication.
    This immediately follows from the distribution law of matrices.
\end{myproof}
\begin{remark}
    \label{rmk:blnrep}
    The above theorem tells us that
    \(\Lsp(V,W;k)\) and \(\Msp_{m,n}(k)\)
    are isomorphic vector spaces;
    thus the dimension of \(\Lsp(V,W;k)\) is precisely \(nm\),
    the product of dimensions of \(V\) and \(W\).
    However, note that the construction of \(B\) depends on
    the choice of bases \(\mathfrak B\) and \(\mathfrak C\).
    Hence, it is dangerous to say that
    \(\Lsp(V,W;k)\) and \(\Msp_{m,n}(k)\) are
    \emph{naturally} isomorphic vector spaces
    as if \(V\iso V^{\dast}\) for
    a finite-dimensional vector space \(V\).
\end{remark}