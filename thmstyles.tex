% restate theorem
\usepackage{thm-restate}

% framed styles for important theorems
\usepackage{framed}
\colorlet{rulecolor}{Navy}
\colorlet{headcolor}{Navy}
\renewenvironment{leftbar}{%
  \def\FrameCommand{{\color{rulecolor}\vrule width 3pt}\hspace{0.5em}}%
  \MakeFramed {\advance\hsize-\width \FrameRestore}}%
 {\endMakeFramed}

% theorem styles
\declaretheoremstyle[
    headfont={\color{headcolor}\sffamily\bfseries},
    bodyfont=\normalfont,
    notefont=\sffamily\mdseries,
    notebraces={(}{)},
    headformat={\NAME\space\NUMBER.\NOTE},
    headpunct={},
    preheadhook=\vspace{0.5\baselineskip},
    postfoothook=\vspace{0.5\baselineskip}
]{lightthm}

\declaretheorem[
    style=lightthm,
    name=Definition,
    refname={Definition,Definitions},
    within=section,
    postheadhook=\begin{leftbar},
    prefoothook=\end{leftbar},
]{definition}
\declaretheorem[
    style=lightthm,
    name=Definition,
    refname={Definition,Definitions},
    unnumbered,
]{definition*}
\declaretheorem[
    style=lightthm,
    name=Example,
    refname={Example,Examples},
    sibling=definition
]{example}
\declaretheorem[
    style=lightthm,
    name=Exercise,
    refname={Exercise,Exercises},
    sibling=definition
]{exercise}
\declaretheorem[
    style=lightthm,
    name=Notation,
    refname={Notation,Notations},
    sibling=definition,
    postheadhook=\begin{leftbar},
    prefoothook=\end{leftbar},
]{notation}
\declaretheorem[
    style=lightthm,
    name=Caution,
    refname={Caution,Cautions},
    sibling=definition
]{caution}
\declaretheorem[
    style=lightthm,
    name=Remark,
    refname={Remark,Remarks},
    sibling=definition
]{remark}
\declaretheorem[
    style=lightthm,
    name=Redefinition,
    refname={Redefinition,Redefinitions},
    sibling=definition,
    postheadhook=\begin{leftbar},
    prefoothook=\end{leftbar},
]{redefinition}
% \declaretheorem[
%     style=lightthm,
%     name=전환점,
%     refname={전환점},
%     sibling=definition
% ]{turnpoint}
\declaretheorem[
    style=lightthm,
    name=Convention,
    refname={Convention,Conventions},
    sibling=definition,
    postheadhook=\begin{leftbar},
    prefoothook=\end{leftbar},
]{convention}

\declaretheoremstyle[
    headfont={\color{headcolor}\sffamily\bfseries},
    bodyfont=\sffamily,
    notefont=\sffamily\bfseries,
    notebraces={(}{)},
    headformat={\NAME\space\NUMBER.\NOTE},
    headpunct={},
    preheadhook=\vspace{0.5\baselineskip},
]{normalthm}

\declaretheorem[
    style=normalthm,
    name=Observation,
    refname={Observation,Observations},
    sibling=definition,
    postheadhook=\begin{leftbar},
    prefoothook=\end{leftbar},
]{observation}
\declaretheorem[
    style=normalthm,
    name=Observation,
    refname={Observation,Observations},
    unnumbered,
]{observation*}
\declaretheorem[
    style=normalthm,
    name=Proposition,
    refname={Proposition,Propositions},
    sibling=definition,
]{proposition}
\declaretheorem[
    style=normalthm,
    name=Lemma,
    refname={Lemma,Lemmas},
    sibling=definition,
    postheadhook=\begin{leftbar},
    prefoothook=\end{leftbar},
]{lemma}
\declaretheorem[
    style=normalthm,
    name=Theorem,
    refname={Theorem,Theorems},
    sibling=definition,
    postheadhook=\begin{leftbar},
    prefoothook=\end{leftbar},
]{theorem}
\declaretheorem[
    style=normalthm,
    name=Corollary,
    refname={Corollary,Corollaries},
    sibling=definition,
]{corollary}
\declaretheorem[
    style=normalthm,
    name=Question,
    refname={Question,Questions},
    sibling=definition,,
    postheadhook=\begin{leftbar},
    prefoothook=\end{leftbar},
    postfoothook=\vspace{0.5\baselineskip}
]{question}

\declaretheoremstyle[
    headfont=\sffamily\bfseries,
    bodyfont=\normalfont,
    headpunct={\::}
    ]{sketchproof}

\declaretheorem[
    style=sketchproof,
    name=Sketch of Proof,
    unnumbered
]{sketch}

\declaretheoremstyle[
    headfont=\sffamily\bfseries,
    bodyfont=\normalfont,
    qed=\qedsymbol,
    headpunct={\::},
    postfoothook=\vspace{0.5\baselineskip}
]{proof}
    
\declaretheorem[
    style=proof,
    name=Proof,
    unnumbered,
]{myproof}
\declaretheorem[
    style=proof,
    name=Solution,
    unnumbered,
]{solution}
\newcommand\rightqed{{\hfill\qedhere}} % place qed at right side

\declaretheoremstyle[
    headfont=\sffamily\bfseries,
    notefont=\sffamily\bfseries,
    headformat={\NOTE}, % note becomes the title of the theorem.
    headpunct={\::},
    notebraces={\unskip}{},
    qed=\qedsymbol
]{miscellaneous}

\declaretheorem[
    style=miscellaneous,
    unnumbered,
    postfoothook=\vspace{0.5\baselineskip}
]{misc}